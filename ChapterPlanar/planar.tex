\chapter{Planarity and more}
\section{The Plane}
\lettrine[lines=4]{T}{he Plane}, is simply two-dimensional space. You may also see it refered to as $\mathbb{R}^2$ or the Euclidean Plane.\newline
\subsection{Plane and Simple}
A graph is \textbf{planar} if it can be embedded in the plane without without any edges crossing.\newline
A \textbf{plane graph} is a graph that has been drawn without any edges crossing.\newline
\begin{figure}[!ht]
	\centering
	\subfloat[Planar]{\includegraphics[width=0.45\textwidth]{ChapterPlanar/k4.pdf}}
	\subfloat[Plane]{\includegraphics[width=0.45\textwidth]{ChapterPlanar/k4Plane.pdf}}
\end{figure}

\section{Don't cross me}
The \textbf{Crossing Number} of a graph G, denoted "cr(G)", is the smallest number of edge crossings needed to draw the graph in the plane.\newline
All planar graphs have a crossing number of 0. In the figure below, you can see that both $K_5$ and $K_{3,3}$ have crossing numbers of 1. (there isn't a figure yet get \textit{allll} the way off my back about it)
%put a figure here

\section{Jordan Curves}
\section{Faces}
\subsection{Euler's formula}
Let \textbf{G} be a connected plane graph and let $n$, $m$, and $f$ be the number of vertices, edges, and faces of \textbf{G} (respectively).
\begin{equation*}
	n-m+f=2
\end{equation*}
\section{We're just Platonic}
The \textbf{Platonic Solids} are the five 3-dimensional solids whose faces consist only of regular congruent polygons.\newline
These solids are (in order of face count increasing): the Tetrahedron, the Cube, the Octahedron, the Dodecahedron, and the Icosahedron.\newline %And the Utah Teapot
%image of the solids goes here
The \textbf{Platonic Graphs} are the graphs of the platonic solids. They are all planar.
